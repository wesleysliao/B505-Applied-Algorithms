HW1 B505 Applied Algorithms

2020-09-05
Wesley Liao


1.

A = 31, 41, 59, 26, 41, 58

i = 1
41 > 31
31, 41, 59, 26, 41, 58

i = 2
59 > 41
31, 41, 59, 26, 41, 58

i = 3
26 < 59
26 < 41
26 < 31
26, 31, 41, 59, 41, 58

i = 4
41 < 59
41 = 41
26, 31, 41, 41, 59, 58

i = 5
58 < 59
58 > 41
26, 31, 41, 41, 58, 59


2.
 (1) The algorithm will return the greatest number in the input array.
 (2) O(n)


3.
((n/2) - 1) * (n - 2))
O(n^2)


4.
f(n) <= c * g(n)
f(n) = 1/n
g(n) = 1
1/n <= c * 1
true as long as c > 1/n_0 and n_0 > 0

5.


6.
(n, log n)

    f(n) = O(g(n))
    f(n) <= c * g(n)

    log n < c * n
    log n = O(n)


(n^2, 2^n)

    f(n) = O(g(n))
    f(n) <= c * g(n)

    n^2 <= c * 2^n
    true if c >= 1 and n >= 4


(2^n, 3^n)

    f(n) = O(g(n))
    f(n) <= c * g(n)

    2^2 <= c * 3^n
    true if c >= 1 and n >= 0


(log n, log^2 n)

    f(n) = O(g(n))
    f(n) <= c * g(n)

    log n  <= c * log^2 n
    true if c >= 1 and n >= 2


7.
A = array of length N
for i = 1 to N:
    k = random integer equal or greater to i and less than N
    swap A[i] with A[k]
